\documentclass{article}
\usepackage[margin=1in]{geometry}
\usepackage{amsfonts}
\usepackage{amsmath}
\usepackage{amssymb}
\usepackage{amsthm}
\usepackage{braket}
\usepackage{enumitem}
\usepackage{mathtools}
\usepackage{parskip}

\newcommand{\R}{\mathbb{R}}
\newcommand{\C}{\mathbb{C}}

\begin{document}

\title{\vspace{-2cm}MTH 532 Homework 8}
\author{Roy Howie}
\date{March 23, 2017}
\maketitle

\section*{2.4 Intersection Theory Mod 2}
  \subsection*{Exercise 1}
    Let $f\colon\C\to\C$ be defined by $z\mapsto z^7+\cos(|z|^2)(1+93z^4)$. To
    show there is a $z\in\C$ such that $f(z)=0$, note that $f(-10)<0<f(10)$. As
    $f$ is continuous, by the Intermediate Value Theorem, there is an $z\in\C$
    such that $f(z)=0$. (I suspect the book wanted us to construct a homotopy
    between $z^7$ and $f$, then consider $\deg_2(f)$ over a sufficiently large
    disk.)
    \qed

  \subsection*{Exercise 11}
    Suppose $f\colon X\to Y$ is not surjective, then there is a point $y$
    outside the image of $f$. Note $y$ is transversal to $f(X)$, as there is no
    intersection, impying $I_2(f, \set{y})=0$. But $Y$ is connected, so there is
    a homotopy between $y$ and some other point $y_0$ contained in the image of
    $f$. As homotopic maps have the same degree mod 2, $\deg_2(f)=0$. Thus, for
    surjective maps, $\deg_2(f)=1$.
    \qed

  \subsection*{Exercise 12}
    Use exercise \textbf{2.4.11}. In this case, a map $f\colon X\to Y$ cannot be
    surjective, so it must be that $\deg_2(f)=0$.
    \qed

\section*{3.2 Orientation}
  \subsection*{Exercise 2}
    \begin{enumerate}[label=\textbf{(\alph*)}]
      \item{
        Note in this case the linear isomorphism $A$ from $V\to V$ which sends
        $\beta$ to $\beta'=\set{v_1,\cdots,cv_i,\cdots,cv_k}$ is the identity
        matrix with the entry at row and column $i$ replaced by $c$. Hence,
        $\det A=c$, so $\beta'$ is equivalently oriented iff $c>0$ and $\beta'$
        is oppositely oriented iff $c<0$.
      }
      \item{
        Let $E_{i,j}$ be the elementary row operator matrix which switches row
        $i$ with row $j$. Note $\det E_{i,j}=-1$ and that $E_{i,j}$ performs the
        mentioned change of ordered basis. Hence, transposing two elements
        produces an oppositely oriented basis.
      }
      \item{
        Again, as in \textbf{(b)}, this is another elementary row operation
        matrix, but with determinant one, thus producing an equivalently
        oriented basis.
      }
      \qed
    \end{enumerate}

  \subsection*{Exercise 4}
    Use exercise \textbf{3.2.2b}. Let $\dim V_1=a$ and $\dim V_2=b$, with
    ordered bases $\beta_1=\set{x_1,\cdots,x_a}$ and $\beta_2=\set{y_1,\cdots,
    y_b}$, respectively. Then $V_1\oplus V_2$ has an ordered basis $B=\set{x_1,
    \cdots,x_a,y_1,\cdots,y_b}$. Similarly, $V_2\oplus V_1$ has an ordered basis
    $B'=\set{y_1,\cdots,y_b,x_1,\cdots,x_a}$. Hence, we can construct a linear
    map which is the composition of the elementary row operator which switches
    $v_i$ with $v_{i+1}$. To go from $B$ to $B'$, this would have to be applied
    $b$ times for each element in $V_1$, i.e. $ab$ times. [Start with $x_a$
    and move it ``to the end,'' then do $x_{a-1}$, etc.]

    Thus, the determinant of this linear map is $(-1)^{ab}=(-1)^{(\dim V_1)(\dim
    V_2)}$.
    \qed

  \subsection*{Exercise 6}
    Let $\beta=\set{v_1,\cdots,v_{k-1}}$ be an ordered basis for $\delta H^k$
    and let $n$ be the outward facing normal. Then $\beta'=\set{n,\beta}$ has
    the same sign as $\beta$. Note that $\beta'$ is oppositely oriented compared
    to the standard ordered basis on $\R^k$. We can then ``move'' $n$ to the end
    of the basis, which changes the sign by $(-1)^{k-1}$. We can then change $n$
    to $-n$. Thus, for even $k$, the sign does not change, as desired, for we
    found the opposite orientation of $\R^{k-1}$.
    \qed


\end{document}
