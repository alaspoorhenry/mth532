\documentclass{article}
\usepackage[margin=1in]{geometry}
\usepackage{amsfonts}
\usepackage{amsmath}
\usepackage{amssymb}
\usepackage{parskip}
\usepackage{enumitem}
%\usepackage{tkz-graph}
%\usetikzlibrary{arrows}

\begin{document}

\title{MTH 532 Homework 3}
\author{Roy Howie}
\date{February 9, 2017}
\maketitle

\subsection*{Exercise 3}
\subsection*{Exercise 4}
\subsection*{Exercise 6}
\begin{enumerate}[label=(\alph*)]
\item{}
\item{}
\item{}
\item{}
\end{enumerate}

\subsection*{Exercise 7}
\begin{enumerate}[label=(\alph*)]
\item{
	Let $g\colon\mathbb{R}^1\to S^1$ be defined by
	$t\mapsto(\cos2\pi t,\ \sin2\pi t)$, then
	$dg_t = (-2\pi\sin2\pi t,\ 2\pi\cos2\pi t)$.
	Since $dg_t\ne(0,0)$ for all $t$, by the
	Inverse Function Theorem, $g$ is a local
	diffeomorphism.
}
\item{
	Let $G=g\times g\colon L\to S^1\times S^1$
	be defined by $(a,b)\mapsto(\cos2\pi a,\ \sin2\pi a,
	\cos2\pi b,\ \sin2\pi b)$, where $L\subset\mathbb{R}^2$
	is a line of irrational slope. Without loss of generality,
	suppose $L$ has no constant term, as that only serves to ``shift''
	the image of $L$ about the torus. That is, let $i\in\mathbb{R}-\mathbb{Q}$
	and define $L=\{(x,y)\mid y=ix\}$.
	
	Next, suppose $G(s,is)=G(t,it)$. Then $\cos2\pi s=\cos2\pi t$,
	so $s-t\in\mathbb{Z}$. Similarly, $\cos2\pi is=\cos2\pi it$,
	so $is-it=i(s-t)\in\mathbb{Z}$. But $i$ is irrational, so
	$i(s-t)$ is in $\mathbb{Z}$ iff $s-t=0$. Thus, $G$ is injective.
	\hfill $\square$
}
\end{enumerate}

\subsection*{Exercise 8}
Let $h\colon\mathbb{R}^1\to\mathbb{R}^2$ be defined by
$t\mapsto\frac{1}{2}(e^t+e^{-t},e^t-e^{-t})$.
To show $h$ is an embedding, we must show that it is
an immersion, injective, and proper. Note that
$dh_t= \frac{1}{2}
\left[
	\begin{smallmatrix}
		e^t-e^{-t}\\
		e^t+e^{-t}
	\end{smallmatrix}
\right]$
is injective for all $t$, as $e^{-t}$ is everywhere nonzero.
Hence, $h$ is an immersion.

Next, suppose $h$ is not injective. Then there exist $a\ne b$
such that $h(a)=h(b)$. But then
\begin{align*}
	e^a+e^{-b} &= e^b+e^{-a}\\
	e^{a+b}(e^a+e^{-b}) &= (e^b+e^{-a})e^{a+b}\\
	e^a(e^{a+b}+1) &= (e^{b+a}+1)e^b\\
	e^a &= e^b
\end{align*}
Since $x\mapsto e^x$ is injective, $a=b$. A contradiction, so
$h$ is injective.

Since $h$ is continuous, the preimage of a closed set is closed.
We need only show the preimage of a bounded set is itself bounded.
Suppose $B\subset\mathbb{R}^2$ is a bounded set. If
$B\cap h(\mathbb{R}^1)=\varnothing$, we're done, as the empty set
is trivially bounded.
Otherwise, let $B_{x,y}(r)$ be the ball of radius $r$ centered at
$(x,y)$ containing $B$. Let $x_0 = h^{-1}(x, h(x))$, then
$h^{-1}(B)$ is bounded by the interval $(x_0-r,x_0+r)$.

Hence, $h$ is proper and thus an embedding.

To show its image is one nappe of the hyperbola $x^2-y^2=1$, consider
\begin{align*}
	\frac{1}{4}(e^t+e^{-t})^2-\frac{1}{4}(e^t-e^{-t})^2 &= 1\\
	(e^{2t}+2+e^{-2t})-(e^{2t}-2+e^{-2t}) &= 4\\
	(e^{2t}-e^{2t})+(e^{-2t}-e^{-2t})+4 &= 4\\
\end{align*}
which clearly checks out.
\hfill $\square$

\end{document}