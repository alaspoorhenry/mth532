\documentclass{article}
\usepackage[margin=1in]{geometry}
\usepackage{amsfonts}
\usepackage{amsmath}
\usepackage{amssymb}
\usepackage{amsthm}
\usepackage{braket}
\usepackage{enumitem}
\usepackage{mathtools}
\usepackage{parskip}

\newcommand{\R}{\mathbb{R}}

\begin{document}

\title{\vspace{-2cm}MTH 532 Homework 7}
\author{Roy Howie}
\date{March 9, 2017}
\maketitle

\section*{2.1 Manifolds with Boundary}
  \subsection*{Exercise 1}
    Let $U\subset\R^k$ and $V\subset H^k$ be open neighborhoods of 0. Note that
    $V-\{0\}$ is simply connected but $U-\{0\}$ is not. Hence, there is no
    diffeomorphism between the two spaces. [Note: in a sense, problem 2.4.10 is
    another example of this and can also be generalized to any dimension.]
    \qed

  \subsection*{Exercise 2}
    Consider a diffeomorphism $f\colon X\to Y$ of manifolds with boundary. Pick
    $p\in\delta X$, then there is an open neighborhood $O(p)$ about $p$ with the
    property that $f|_{O(p)}$ is a linear isomorphism. Furthermore, $O(p)$ can
    be parameterized by $U\subset H^k$ and $f|_{O(p)}$ is equivalent to $id
    \colon U\to U$. It follows from this parameterization that $x\in\delta X$
    iff $\delta f(x)\in\delta Y$.
    \qed

  \subsection*{Exercise 4}
    Use the lemma from page 62. Consider the function $\pi\colon\R^3\to\R$
    defined by $(x,y,z)\mapsto a-x^2-y^2+z^2$. Since $x^2+y^2-z^2\leq a$, we
    have that $\pi(x)\geq0$ for all $x\in H$, where $H$ is the given solid
    hyperboloid. Thus, by the lemma, $\pi^{-1}([0,\infty])=H$ is a manifold with
    boundary.
    \qed

  \subsection*{Exercise 6}
    Not sure how to ``check,'' other than to draw a single line around the
    entire Möbius band without having to lift one's pen. Naturally, this doesn't
    work for the cylinder.
    \qed?

    If the strip is twisted $n$ times before gluing, then the resulting object
    is diffeomorphic to the cylinder iff $n$ is even and diffeomorphic to the
    Möbius band iff $n$ is odd.
    \qed

\section*{2.2 One-Manifolds and Some Consequences}
  \subsection*{Exercise 3}
    A map which rotates the torus about its center by $\pi$ has no fixed points.
    The proof of the Brouwer theorem fails because the solid torus has a hole in
    its center, so the retraction to its boundary doesn't work.
    \qed

  \subsection*{Exercise 4}
    Per the hint, consider the map from the open ball $f\colon B(0,a)\to\R^k$
    specified in problem 1.1.4. Next, consider a continuous function $g\colon
    \R^k\to\R^k$ with no fixed point, say $x\mapsto (x_1+1,x_2,\cdots,x_k)$.
    Note that $f^{-1}$ exists. Thus, we have that $f^{-1}\circ g\circ f$ is a
    map from $B(0,1)$ to itself with no fixed point.
    \qed

\section*{2.4 Intersection Theory Mod 2}
  \subsection*{Exercise 7}
    Per the hint, note that $\deg_2(id)\equiv1$. However, if $S^1$ were simply
    connected, then every map $f\colon S^1\to S^1$ would be homotopic to a
    constant map. But $\deg_2(f)\equiv0$ and homotopic maps must have the same
    degree modulo 2. This is a contradiction, so $S^1$ is not simply connected.
    \qed

  \subsection*{Exercise 10}
    Note that $T^2=S^1\times S^1$. Consider the simple, closed paths $S^1\times
    \{a\}$ and $\{b\}\times S^1$, where $a,b\in S^1$. Note that they intersect
    at a single point: $(a,b)\in T^2$.

    On the other hand, consider two simple, closed, nonintersecting paths on
    $S^2$. From problem 1.7.6, since $S^2$ is simply connected for $k>1$, these
    paths are each contractible to a single point.

    If there were a diffeomorphism between $S^2$ and $T^2$, this would be a
    contradiction, as they each have different intersection numbers modulo 2.
    \qed

\end{document}
