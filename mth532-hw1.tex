\documentclass{article}
\usepackage[margin=1in]{geometry}
\usepackage{amsfonts}
\usepackage{amsmath}
\usepackage{amsthm}
\usepackage{parskip}
\usepackage{enumitem}

\newcommand{\N}{\mathbb{N}}
\newcommand{\R}{\mathbb{R}}

\begin{document}

\title{MTH 532 Homework 1}
\author{Roy Howie}
\date{January 30, 2017}
\maketitle

\subsection*{Exercise 1}
  Let $k < l$ and let $f$ be a smooth function on $\R^k$. Let $F$ be a function
  on $\R^l$ such that $F(a_1,\cdots,a_k,0,\cdots,0)=f(a_1,\cdots,a_k)$. Consider
  $\pi \colon \ \R^k \to \R^k$ defined by $x \mapsto x$ and note that $F=f\circ
  \pi$. Hence, $F$ is the composition of smooth functions and is therefore
  smooth itself.

  On the other hand, suppose $F$ is smooth on $\{(a_1,\cdots,a_k,0,\cdots,0)\}$.
  Let $i \colon \ \R^k \to \R^l$ be the smooth map defined by $(a_1,\cdots,a_k)
  \mapsto (a_1,\cdots,a_k,0,\cdots,0)$ and let $f$ be a function on $\R^k$ such
  that $f(a_1,\cdots,a_k)=F(a_1,\cdots,a_k,0,\cdots,0)$. Then $F$ is smooth, as
  it is the composition of the smooth maps $f$ and $i$.
  \qed

\subsection*{Exercise 2}
  Let $f$ be smooth on $X \subset \R^N$. As $f$ is smooth on $X$, for every $p
  \in X$, there is an open neighborhood $O(p)\subset\R^N$ such that $F\colon
  O(p)\to\R$ is smooth on $O(p)$ and $F(p)=f(p)$ for all $p\in O(p)\cap X$.

  Now consider $Z\subset X$ and, for every $p\in Z$, take $O(p)$ and $F$ as
  before. Then $F(p)=f(p)$ for all $p\in O(p)\cap Z$ and $f$ is thus smooth on
  $Z$.
  \qed

\subsection*{Exercise 4}
  \begin{enumerate}[label=\textbf{\alph*}]
    \item{
      Let $f^{-1} \colon \ \R^k \to B_a$ be the smooth map defined by
      $$f^{-1}(y) = \frac{ay}{\sqrt{a^2 + \vert\vert y \vert\vert^2}}$$
      Note that $f^{-1}(0)=0$ and $\lim_{y\to \infty}f^{-1}(y)=a$. That is,
      $f^{-1}$ maps $[0,\infty)^k$ to $[0,a)^k$, which makes intuitive sense as
      $f$ did the opposite.
    }
    \item{
      Since $X$ is a manifold, for every $x \in X$, there is a parameterization
      $p\colon\ U\to O(x)$ where $U\subset\R^k$ and $O(x)$ is an open
      neighborhood of $x$. But $U$ can be the ball $B_a$ of radius $a$, as there
      is always one small enough inside of $U$ such that $x\in p(B_a)\subset V$.
      So consider $f^{-1}\circ p$ restricted to $B_a$, which is a
      parameterization of an open neighborhood of $x$ with all of $\R^k$ as its
      domain.
    }
    \qed
  \end{enumerate}

\subsection*{Exercise 6}
  Let $h(x)=x^{1/3}$. Note that $f\circ h=h\circ f=id$ and that $h'(x)=
  \frac{1}{3}x^{-2/3}$. However $\lim_{x\to0}h(x)$ does not exist, so $h$ is not
  smooth and $f$ is not a diffeomorphism.
  \qed

\subsection*{Exercise 8}
  Let $a>0$ and let $H$ be the hyperboloid $\{(x,y,z)\mid x^2+y^2-z^2=a\}$. Let
  $B_a$ be the ball of radius $a$ centered at the origin. The upper half of $H$
  can then be parameterized via $\phi\colon\ \R^2-B_a\to\R^3$ defined by $(x,y)
  \mapsto(x,y,\sqrt{x^2+y^2-a})$. Similarly, the lower half of $H$ can be
  parameterized by $(x,y)\mapsto(x,y,-\sqrt{x^2+y^2-a})$. Intuitively speaking,
  this involves lifting the plane minus $B_a$ so that it ``covers'' the given
  half of $H$.

  When $a=0$, the point $(0,0,0)$ becomes a problem. Removing the origin from
  $\R^2$ leaves one component, whereas removing the origin from $H$ leaves two
  components, so $H$ is not a manifold.
  \qed

\subsection*{Exercise 12}
  Let $N=(0,0,1)$ and let $p$ be a point on $S^2$. The line through points $N$
  and $p$ then has the equation
  \begin{align*}
    l(t) &=(0+t(x-0),\ 0+t(y-0),\ 1+t(z-1))\\
         &=(tx,\ ty,\ 1+t(z-1))
  \end{align*}
  This line hits the xy-plane when $z=0$, or when $1+t(z-1)=0$, implying $t=
  \frac{1}{1-z}$. Hence,
    $$\pi(x,y,z) = (\frac{x}{1-z},\frac{y}{1-z})$$
  To find $\pi^{-1}$, note that
    \begin{align*}
      \pi^{-1}(0,0) &= -N\\
      \pi^{-1}(1,0) &= (1,0,0)\\
      \pi^{-1}(0,1) &= (0,1,0)\\
    \end{align*}
  and $\vert\vert(x,y)\vert\vert=1\iff z=0$. I couldn't think of a function $z=
  f(x,y)$ which satisfied these conditions, but google gave me $z=
  \frac{x^2+y^2-1}{x^2+y^2+1}$, which definitely works. This makes finding
  $\pi^{-1}$ easy:
    $$
      \pi^{-1}(x,y) = (
        \frac{2x}{1+x^2+y^2},
        \frac{2y}{1+x^2+y^2},
        \frac{x^2+y^2-1}{1+x^2+y^2}
      )
    $$
  \qed

\subsection*{Exercise 14}
  Let $(x,y)\in X\times Y$ and let $U\times V$ be an open neighborhood of
  $(x,y)$ such that $F$ is smooth on $U$, $G$ is smooth on $V$, $F$ restricted
  to $U\cap X$ equals $f$, and $G$ restricted to $V\cap Y$ equals $g$. Note that
  $(U\times V)\cap(X\times Y)=(U\cap X)\times(V\cap Y)$. Hence, since $F\times
  G$ is smooth on $(U\cap X)\times(V\cap Y)$, it is also smooth on $(U\times V)
  \cap(X\times Y)$, so $f\times g$ is too.
  \qed

\subsection*{Exercise 18}
  \begin{enumerate}[label=\textbf{\alph*}]
    \item{
      From class, we had that $f^{(n)}(x)=P_ne^{-1/{x^2}}$, where $P_n$ is a
      polynomial of order $n$ or less. Thus $\lim_{x\to0}f^{(n)}(x)=0$ for all
      $n\in\N$, so $f$ is smooth.
    }
    \item{
      Subtraction and $(x,y)\mapsto xy$ are smooth functions, so $g$, the
      composition of smooth functions, is too. Since $g$ is smooth and positive
      function on $(a,b)$, we have that $c=\int_{-\infty}^{\infty}g\ dx$ is
      nonzero. Hence, $h(x)=\frac{1}{c}\int_{-\infty}^{x}g\ dx=\frac{1}{c}G(x)$
      by the Second Fundamental Theorem of Calculus. As $G'=g$ and $g$ was
      smooth, $h$ must be too, with $h^{(n)}=\frac{1}{c}G^{(n)}$ for all
      $n\in\N$.
    }
    \item{
      Consider the function $r(x)=1-h(\vert\vert x\vert\vert)$.
    }
    \qed
  \end{enumerate}

\end{document}
