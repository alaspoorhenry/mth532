\documentclass{article}
\usepackage[margin=1in]{geometry}
\usepackage{amsfonts}
\usepackage{amsmath}
\usepackage{amssymb}
\usepackage{amsthm}
\usepackage{braket}
\usepackage{enumitem}
\usepackage{mathtools}
\usepackage{parskip}

\newcommand{\R}{\mathbb{R}}
\newcommand{\wprod}[2]{\,#1\wedge #2}
\newcommand{\wprods}[3]{\,#1\wedge #2\wedge #3}

\DeclareSymbolFont{matha}{OML}{txmi}{m}{it}% txfonts
\DeclareMathSymbol{\varv}{\mathord}{matha}{118}

\begin{document}

\title{\vspace{-2cm}MTH 532 Homework 12}
\author{Roy Howie}
\date{April 20, 2017}
\maketitle

\section*{4.5 Exterior Derivative}
  \subsection*{Exercise 1}
    \begin{enumerate}[label=\textbf{(\alph*)}]
      \item{
        Let $\omega=z^2\,dx\wedge dy+(z^2+2y)\,dx\wedge dz$, then $d\omega=
        (2z-2)\wprods{dx}{dy}{dz}$.
      }
      \item{
        Let $\omega=13x\,dx+y^2\,dy+xyz\,dz$, then $d\omega=xz\wprod{dy}{dz}-yz
        \wprod{dz}{dx}$.
      }
      \item{
        If $f,g$ are functions, then $d(fdg)=dfdg+f(ddg)$, but $d(dg)=0$, so
        $d(fdg)=dfdg=\nabla f\nabla g$.
      }
      \item{
        Let $\omega=(x+2y^3)(\wprod{dz}{dx}+\frac{1}{2}\wprod{dy}{dx})$, then
        $d\omega=6y^2\wprods{dx}{dy}{dz}$.
      }
      \qed
    \end{enumerate}
  \subsection*{Exercise 2}
    Let $F=\left(\frac{-y}{x^2+y^2},\,\frac{x}{x^2+y^2},\,0\right)$, then
    $\text{curl}\,F=\nabla\times F=\left(0,0,\frac{\delta}{\delta x}\frac{x}{x^2
    +y^2}-\frac{\delta}{\delta y}\frac{-y}{x^2+y^2}\right)=\left(0,0,
    \frac{y^2-x^2}{(x^2+y^2)^2}+\frac{x^2-y^2}{(x^2+y^2)^2}\right)=\vec{0}$. It
    cannot be written as the gradient of any function because derivatives are
    unique and the gradient of $z=\arctan(y/x)$ restricted to $x$ and $y$
    produces the given vector field, as found in exercise \textbf{4.4.8b}.

\section*{4.7 Stokes Theorem}
  \subsection*{Exercise 6}
    Let $D$ be a compact region in $\R^3$ with smooth boundary $S$. Let $D'=D-
    B_0$, where $B_0$ is a ball of infinitesmal radius centered at the origin.
    Let $\vec{F}=q\vec{r}/r^3=q(x,y,z)(x^2+y^2+z^2)^{-3/2}$. Note that
    $\text{div}(\vec{F})=0$ for all $r\ne0$. Denote $\delta B_0=S_R$, a sphere
    of radius $r$. Then,
    \begin{align*}
      \int_S \vec{F}\cdot\vec{n}\,dA
        &= \int_{D'}\text{div}(\vec{F})\,dV\\
        &= \int_D 0\,dV - \int_{B_0} \text{div}(\vec{F})\,dV\\
        &= \int_{S_R}\vec{F}\cdot\vec{n}\,dA\\
        &= \int_{S_R} (q\vec{r}/r^3)\cdot(\vec{r}/r)\,dA\\
        &= q/r^2\int_{S_R}\,dA=4\pi q
        \tag*{\qed}
    \end{align*}

  \subsection*{Exercise 8}
    Let $X=\delta W$ with $W$ compact and let $f\colon X\to Y$ be a smooth map.
    Let $\omega$ be a closed $k$-form on $Y$ with $k=\dim X$. Suppose $f$
    extends to all of $W$. Note that $W$ has dimension $l=k+1$, so $\omega$ is a
    $l-1$ form. Apply the generalized Stokes theorem:
    \begin{equation*}
      \int_X f^*\omega
        = \int_{W} d(f^*\omega)
        = \int_{W} f^*(d\omega)
        = 0
    \end{equation*}
    This follows from the fact that $\omega$ is closed on $Y$.
    \qed

  \subsection*{Exercise 9}
    Let $f_0,f_1\colon X\to Y$ be smooth homotopic maps and let $X$ be a smooth,
    boundaryless manifold of dimension $k$. Suppose $\omega$ is a closed
    $k$-form on $Y$. Let $W=X\times[0,1]$ and consider the homotopy $F\colon W
    \to Y$ with $F(x,0)=f_0(x)$ and $F(x,1)=f_1(x)$. Note $W$ is a manifold with
    boundary, so, by exercise \textbf{4.7.8}, we have that $\int_{\delta W}F^*
    \omega=0$. On the other hand, we have
    \begin{equation*}
      \int_{\delta W}F^*\omega
        =\int_{X\times[0,1]}F^*\omega
        =\int_{X\times\set{0}}F^*\omega + \int_{-X\times\set{1}}F^*\omega
        =\int_{X}f_0^*\omega-\int_{X}f_1^*\omega
    \end{equation*}
    Thus, $\int_Xf_0^*\omega=\int_Xf_1^*\omega$.
    \qed

  \subsection*{Exercise 10}
    Let $X$ be a simply connected manifold, $\omega$ be a closed 1-form on $X$,
    and $\gamma$ be a closed curve in $X$. Note that $\gamma$ can contracted to
    a single point, as $X$ is simply connected. Fix $c\in X$ and let $f_c\colon
    S^1\to X$ be defined by $*\mapsto c$. Consider the homotopy between $\gamma$
    and $f_c$, then, $\int_{S^1}\gamma^*\omega=\int_{S^1}f_c^*\omega$ by
    exercise \textbf{4.7.9}. Hence,
    \begin{equation*}
      \oint_\gamma\omega
        =\int_{S^1}\gamma^*\omega
        =\int_{S^1}f_c^*\omega
        = 0
    \end{equation*}
    This follows from \textbf{4.7.8}, as $f_c$ extends to all of $D^2$ and
    $\delta D^2=S^1$, implying $\int_{\delta D^2}f_c^*\omega=0$.
    \qed

\end{document}
