\documentclass{article}
\usepackage[margin=1in]{geometry}
\usepackage{amsfonts}
\usepackage{amsmath}
\usepackage{amsthm}
\usepackage{enumitem}
\usepackage{parskip}
\usepackage{tkz-graph}
\usetikzlibrary{arrows}

\newcommand{\R}{\mathbb{R}}

\begin{document}

\title{MTH 532 Homework 2}
\author{Roy Howie}
\date{February 2, 2017}
\maketitle

\subsection*{Exercise 1}
  Let $f\colon U\to X$ and $g\colon V\to Y$ be parameterizations, let $X\subset
  Y$ be a submanifold, and let $i\colon X\to Y$ be an inclusion map. This
  produces the following commutative diagram:
  \begin{center}
    \begin{tikzpicture}
      % unit
      \SetGraphUnit{3}
      % style for vertex
      \GraphInit[vstyle=Empty]
      \tikzset{VertexStyle/.append style = {shape=rectangle,inner sep=0pt}}
      % vertices
      \Vertex[L=$U$]{1}
      \EA[unit=3,L=$V$](1){2}
      \NO[unit=2,L=$X$](1){4}
      \NO[unit=2,L=$Y$](2){3}
      % edges and labels
      \begin{scope}[every node/.style={midway},>=latex']
        \draw[->]   (1)--(2) node [above] {$h$};
          \draw[->]   (4)--(3) node [below] {$i$};
          \draw[->]   (2)--(3) node [right] {$g$};
          \draw[->]   (1)--(4) node [left]  {$f$};
      \end{scope}
    \end{tikzpicture}
  \end{center}
  Note that $h=g^{-1}\circ i\circ f$. However, $h=g^{-1}\circ f$ too, as $i$ is
  the inclusion map bringing $X$ into $Y$. Without loss of generality and to
  simplify notation, let $f(0)=g(0)=x\in X$. Since $g^{-1}\circ i\circ f g^{-1}
  \circ f$, taking the derivative on both sides yields $d(g^{-1}\circ i\circ f)=
  d(g^{-1}\circ f)$, or that
    $dg^{-1}_0\circ di_x\circ df_0=dg^{-1}_0\circ df_0$.
  Moving terms gives $di_x=(dg_0\circ dg^{-1}_0)\circ(df_0\circ df^{-1}_0)$.
  This is the identity map, so $di_x$ is indeed an inclusion map.
  \qed

\subsection*{Exercise 3}
  Let $V$ be a vector subspace of $\R^N$ with dimension $n$. Then there exists a
  parameterization $f\colon\R^n\to V$. But $f$ is linear, so $df=f$. Thus,
  $T_x(V)=df(\R^n)=f(\R^n)=V$.
  \qed

\subsection*{Exercise 4}
  Let $f\colon X\to Y$ be a diffeomorphism, then there is a smooth map $f^{-1}
  \colon Y\to X$ such that $f\circ f^{-1}=Id$. Note $d\ Id=d(f\circ f^{-1})$,
  implying that $Id=df_x\circ df^{-1}_y$ for all $x\in X$ with $y=f(x)$. Hence,
  $df_x$ has an inverse in the form of $df^{-1}_y$, so it is indeed an
  isomorphism of tangent spaces.
  \qed

\subsection*{Exercise 5}
  Let $k\ne l$ and suppose $f\colon\R^k\to\R^l$ is a diffeomorphism. Then $df_x
  \colon T_x(\R^k)\to T_{f(x)}(\R^l)$ is an isomorphism of tangent spaces. Note
  $T_x(\R^k)=\R^k$ and $T_{f(x)}(\R^l)=\R^l$ for all $x$. But $\text{dim}\ \R^k=
  k\ne l=\text{dim}\ \R^l$, meaning $df$ is not bijective. But $df=f$, so $f$ is
  also not bijective. Contradiction! Hence, $f$ is not a diffeomorphism.
  \qed

\subsection*{Exercise 8}
  Let $H=\{(x,y,z)\mid x^2+y^2-z^2=a\}$ be a hyperboloid and let $h\colon B_a(0)
  \to H\subset\R^3$ be a local parameterization of $(\sqrt{a},0,0)$, where
  $B_a(0)$ is the ball of radius $a$ centered at the origin. Define $h$ as
  $(u,v)\mapsto(\sqrt{a+u^2-v^2},u,v)$. (All credit to Wolfram for that
  parameterization.) Then $dh$ is the matrix
  \begin{equation*}
    \begin{bmatrix}
      u/x &   -v/x  \\
      1   &   0     \\
      0   &   1
    \end{bmatrix}
  \end{equation*}
  where $x=\sqrt{a+u^2-v^2}$. Therefore, $dh_{(\sqrt{a},0,0)}$ is the matrix
  $\left[
    \begin{smallmatrix}
      0&0\\
      1&0\\
      0&1
    \end{smallmatrix}
  \right]$
  and the tangent space is the span of $dh_{(\sqrt{a},0,0)}$, or the yz-plane.
  \qed

\subsection*{Exercise 9}
  \begin{enumerate}[label=(\arabic*)]
    \item{
      Let $a\colon U\to X$ and $b\colon V\to Y$ be local parameterizations of
      $X$ and $Y$, respectively. Note that $X\times Y=\text{Im}(a\times b)$.
      Hence, $T_{(x,y)}(X\times Y)=\text{Im}\ d(a\times b)_{(0,0)}$, which can
      be rewritten as the image of
      $\left[
        \begin{smallmatrix}
          da_0  & 0   \\
          0     & db_0
        \end{smallmatrix}
      \right]$, or $T_xX\times T_yY$.
    }
    \item{
      Let $f\colon X\times Y\to X$ be the projection map and let $a\colon U\to
      X$ and $b\colon V\to Y$ be local parameterizations. This forms a
      commutative diagram, meaning there is a map from $U\times V$ to $U$,
      $h=a^{-1}\circ f\circ(a\times b)$. For notation and without loss of
      generality, take $(a\times b)(0,0)=(x,y)$. Shuffling terms yields $f=a
      \circ h \circ(a\times b)^{-1}$, implying $df_{(x,y)}=da_0\circ dh_{(0,0)}
      \circ d(a\times b)^{-1}_{(0,0)}$, which is the desired projection.
    }
    \item{
      Use $a$ and $b$ as before and let $h\colon U\to U\times V$ defined by $u
      \mapsto(u,0)$. Fix $y=b(0)$ and let $f\colon X\to X\times Y$ be the
      injective mapping $x\mapsto(x,y)$. Next, consider $f=(a\times b)\circ h
      \circ a^{-1}$. Note that $dh=h$, as $h$ is linear. Therefore, $df_x=
      (da_0\times db_0)\circ h\circ da^{-1}_0$. Thus, $df_x(v)=(v,0)$.
    }
    \item{
      Intuition: project each space and then smoothly map it to $(x,0)$ and
      $(0,y)$ so that the direct sum ``fills'' the ``containing'' space.

      Let $\pi_x\colon X\times Y\to X$ and $\pi_y\colon X\times Y\to Y$ be
      natural projections. Let $i_x\colon X'\to\R^N$ and $i_y\colon Y'\to\R^N$
      be smooth inclusions defined by $x\mapsto(x,0)$ and $y\mapsto(0,y)$,
      respectively. ($\R^N$ is assumed to ``contain'' $X\times Y$.) Then clearly
      $f\times g=i_x\circ f\circ\pi_x+i_y\circ g\circ\pi_y$. This implies
      $d(f\times g)_{(x,y)}=df_x\times dg_y$.
    }
    \qed
  \end{enumerate}

\end{document}
