\documentclass{article}
\usepackage[margin=1in]{geometry}
\usepackage{amsfonts}
\usepackage{amsmath}
\usepackage{amssymb}
\usepackage{amsthm}
\usepackage{braket}
\usepackage{enumitem}
\usepackage{mathtools}
\usepackage{parskip}

\newcommand{\R}{\mathbb{R}}
\newcommand{\C}{\mathbb{C}}
\DeclareMathOperator{\M}{M}

\begin{document}

\title{\vspace{-2cm}MTH 532 Homework 5}
\author{Roy Howie}
\date{February 23, 2017}
\maketitle

\section*{1.5 Transversality}
  \subsection*{Exercise 2}
    \begin{enumerate}[label=\textbf{(\alph*)}]
      \item{
        Yes, this is one of the given examples.
      }
      \item {
        Yes, let $v$ be a vector in the $xy$-plane such that $v$ is not a
        multiple of $(3,2,0)$, then the three vectors $v$, $(3,2,0)$ and $(0,4,
        -1)$ span $\R^3$.
      }
      \item {
        No, because the $z$-dimension of any combination will always be zero.
      }
      \item {
        Since $T_x\R^a=\R^a$, these two spaces are transversal iff $k+l\geq n$.
      }
      \item {
        Tricky. The two spaces span $\R^n$ iff $k\geq n$ or $l\geq n$, as one
        space is a subset of or equal to the other.
      }
      \item {
        Yes: $v=(a,b)\in V$ can be written as the sum of $(b,b)$ and $(a-b,0)$.
      }
      \item {
        Yes. Let $A\in\M(n)$ be a canonical basis element, i.e., only a single
        entry is nonzero and equal to 1. Then we can represent it via the sum
        of a symmetric ($B$) and skew-symmetric ($C$) matrix. Suppose $a_{ij}
        =1$, then let $b_{ij}=b_{ji}=1$, $c_{ij}=1$, and $c_{ji}=-1$, then
        $A=\frac{1}{2}(A+B)$.
      }
      \qed
    \end{enumerate}

  \subsection*{Exercise 4}
    Let $h\in T_yX\cap T_yZ$, then $h$ is tangent to both $X$ and $Z$, meaning
    it is tangent to $X\cap Z$. Hence, it is in the tangent space of $X\cap Z$,
    i.e., $h\in T_y(X\cap Z)$.

    On the other hand, if $X$ and $Z$ are transversal, then $X\cap Z$ is a
    submanifold of both $X$ and $Z$. Consider the natural inclusion maps $f
    \colon X\cap Z\to X$ and $g\colon X\cap Z\to Z$, then $df_p\colon T_p(X\cap
    Z)\to T_pX$ and $dg_p\colon T_p(X\cap Z)\to T_pZ$ imply that
    $T_p(X\cap Z)\subset T_pX$ and $T_p(X\cap Z)\subset T_pZ$.
    Therefore, $T_p(X\cap Z)\subset T_pX\cap T_pZ$.

    Hence, $T_p(X\cap Z)=T_pX\cap T_pZ$.
    \qed

  \subsection*{Exercise 8}
    Since at any point, the hyperboloid and sphere both have tangent planes,
    in order to be transversal, these two planes must not overlap. Otherwise,
    the ambient space cannot be ``filled.''

    Clearly, when $a=1$ and $z=0$, there is a problem, as both objects have
    tangent planes parallel to the $z$-axis. In fact, they are not transversal
    when $a=1$. For $a>1$, the two objects do not intersect, so transversality
    is uninterestingly true. For $a<1$, the intersection is a horizontal circle;
    the tangent planes do not coincide, so the two objects are transversal.
    \qed

  \subsection*{Exercise 9}
    Recall that if an eigenvalue $\lambda=1$, then there is a corresponding
    eigenvector $v$ such that $Av=v$. Hence, if $1$ is not an eigenvalue, then
    $Av\ne v$ for all $v\in V$. Thus $W\cap\Delta=\varnothing$, so
    transversality is vacuously true.

    Conversely, suppose $1$ is an eigenvalue of $A$, then $W\cap\Delta=\Delta$,
    so the tangent space of the intersection cannot span $V\times V$, meaning
    $W$ is not transversal to $\Delta$. Thus, $1$ is not an eigenvalue of $A$.
    \qed

\section*{1.6 Homotopy and Stability}
  \subsection*{Exercise 3}
    Let $x\sim y$ be an equivalence relation on $X$ with $x\sim y$ iff there is
    a smooth curve $f\colon[0,1]\to X$ such that $f(0)=x$ and $f(1)=y$. Per the
    hint, we shall show $\sim$ is equivalent to homotopy.

    Let $F\colon X\times[0,1]\to X$ be a homotopy with $F(c,0)=x$ and $F(c,1)=y$
    for all $c\in X$. Then, if $x\sim y$ via $f\colon[0,1]\to X$, let $F(c,t)$
    equal $f(t)$. Conversely, if $F$ is a homotopy as defined above, then let
    $f(t)=F(c_0,t)$ for any $c\in X$, implying $x\sim y$. Hence, $\sim$ is an
    equivalence relation.

    Suppose $X/{\sim}$ has more than one component, then it can be parameterized
    by disjoint open sets. However, then there are two points $x$ and $y$ which
    cannot be connected. This is a contradiction, as $X$ is connected. Hence,
    $X/{\sim}$ has a single component and the equivalence classes under $\sim$
    are open.
    \qed

  \subsection*{Exercise 4}
    Let $X$ be contractible, $f\colon Y\to X$ be a smooth map of manifolds,
    $c(x)\colon X\to X$ be the constant map defined by $*\mapsto x$, and $F$ be
    a homotopy between $id$, the identity map on $X$, and $c(x)$.

    Next, construct a homotopy $G$ between $Y$ and $X$ defined by
    $G(y,t)=F(f(y),t)$. Note that $G(y,0)=id\circ f(y)=f(y)$ and $G(y,1)=
    c(x)(1)=x$. Thus, any smooth map is homotopic to the constant map and,
    therefore, any two smooth maps of manifolds are homotopic.
    \qed

  \subsection*{Exercise 6}
    This follows from exercise \textbf{1.6.4}. Suppose $f\colon S^1\to X$ is a
    smooth map and suppose $X$ is contractible, then $f$ is homotopic to a
    constant map on $X$. Let $x,y\in X$. Then there is a homotopy $F_1$ between
    $x$ and $id$ and a homotopy $F_2$ between $id$ and $y$. Thus, there is a
    homotopy $F_0$ between $x$ and $y$, so there is a smooth map connecting the
    two points. $x$ and $y$ were arbitrary, so $X$ is connected. Hence, $X$ is
    simply connected.
    \qed

  \subsection*{Exercise 7}
    Consider $S^{2k-1}\subset\R^{2k}\cong\C^k$. We know $z\mapsto ze^{i\pi}$
    smoothly rotates a point $z\in\C$ by $\pi$. Consider $S^{2k-1}$ as a subset
    of $\C^k$, i.e., $\set{(z_1,\cdots,z_k)\mid z_1^2+\cdots+z_k^2=1}$. Then
    there is a smooth homotopy $R\colon S^{2k-1}\times[0,1]\to S^{2k-1}$ defined
    by $(z_1,\cdots,z_k)\mapsto(z_1e^{i\pi{t}},\cdots,z_ke^{i\pi{t}})$ between
    the identity and antipodal maps.
    \qed

  \subsection*{Exercise 9}
    Consider $t>0$ and note that, for $|tx|>2$, one has $p(tx)=0$. Therefore,
    $f_t$ cannot have any of the properties \textbf{(a)} through \textbf{(f)}.
    For \textbf{(d)}, consider $Z=\set{0}$. Since, $f_t\equiv0$ for
    $|x|>2|t|^{-1}$, it is not transversal to $Z$.
    \qed

\end{document}
