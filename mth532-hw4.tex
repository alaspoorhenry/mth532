\documentclass{article}
\usepackage[margin=1in]{geometry}
\usepackage{amsfonts}
\usepackage{amsmath}
\usepackage{amssymb}
\usepackage{parskip}
\usepackage{enumitem}
\usepackage{mathtools}

\newcommand{\N}{\mathbb{N}}
\newcommand{\Z}{\mathbb{Z}}
\newcommand{\Q}{\mathbb{Q}}
\newcommand{\R}{\mathbb{R}}
\newcommand{\C}{\mathbb{C}}
\DeclareMathOperator{\M}{M}
\DeclareMathOperator{\He}{H}
\DeclareMathOperator{\GL}{GL}
\DeclareMathOperator{\SL}{SL}

\begin{document}

\title{MTH 532 Homework 4}
\author{Roy Howie}
\date{February 16, 2017}
\maketitle

\subsection*{Exercise 7}
\subsection*{Exercise 9}
\subsection*{Exercise 12}

\subsection*{Exercise A1}
  Note that $\M(n,\C)\cong\C^{n^2}$ and that $\det\colon\M(n,\C)\to\C$ is a smooth
  function. Since $\C$ is Hausdorff, $0\in\C$ is closed. Likewise, because of
  the continuitity of $\det$, $\det^{-1}(0)$ is closed too. Note that $\GL(n,\C)
  =M(n,\C)-\det^{-1}(0)$, so $\GL(n,\C)$ is open and, therefore, a smooth manifold.

  Furthermore, $\GL(n,\C)$ is a group under matrix multiplication with identity
  $I_n$. Matrix multiplication is smooth, as it is a polynomial function in the
  entries of the product; matrix inversion is smooth because it is a rational
  function (Cramer's rule), as the determinant is non-vanishing in $\GL(n,\C)$.

  $\GL(n,\C)$ is thus a Lie group of dimension $2n^2$, as its basis consists of
  matrices with one element in $\{1,i\}$.
  \hfill $\square$

\subsection*{Exercise A2}
  Consider the map $\det\colon\M(n,\C)\to\C$. Note that $\SL(n,\C)$ is equal to
  the kernel of the determinant, i.e., $\SL(n,\C)=\det^{-1}(1)$. If $1$ is a
  regular value of $\det$, then $\dim\SL(n,\C)=\dim\GL(n,\C)-\dim\C=2n^2-2$.
  Thus, we wish to show that $1$ is a regular value. We do this by showing $0$
  is the only critical value:
  \begin{align*}
    d\det\nolimits_0(h) &= \lim_{t\to0}\frac{\det(0+th)-\det0}{t}\\
                        &= \lim_{t\to0}\frac{\det(th)}{t}\\
                        &= \lim_{t\to0}t^{n-1}\det(h)\\
                        &= 0
  \end{align*}
  Indeed, $0$ is a critical value. Next, consider $A\in\GL(n,\C)$:
  \begin{align*}
    d\det\nolimits_A(A) &= \lim_{t\to 0}\frac{\det(A+tA)-\det(A)}{t}\\
                        &= \lim_{t\to 0}\frac{(1+t)^n\det(A)-\det(A)}{t}\\
                        &= \det(A)\lim_{t\to 0}\frac{(1+t)^n-1}{t}\\
                        &= \det(A)n
  \end{align*}
  Which is nonzero, as $\det(A)\ne0$. Hence, $1$ is a regular value of $\det$.
  \hfill $\square$

\subsection*{Exercise A3}
  Consider the map $f\colon\M(n,\C)\to\He(n)$ defined by $A\mapsto AA^*$,
  where $\He(n)$ is the set of all $n\times n$ Hermitian matrices and $A^*$ is
  the conjugate transpose. Note that $f^{-1}(I_n)$, where $I_n$ is the identity
  matrix, equals $U(n)$.

  Next, denote $AA^*$ as $C$, then $C^*=(AA^*)^*=A^*(A^*)^*=A^*A=C$, so $C$ is
  Hermitian. Then, observe that Hermitian matrices are of the form
  \begin{equation*}
    \begin{bmatrix}
      a       & c       & \cdots & x\\
      \bar{c} & b       & \cdots & y\\
      \vdots  & \vdots  & \ddots & \vdots\\
      \bar{x} & \bar{y} & \cdots & z
    \end{bmatrix}
  \end{equation*}
  Note that there are $n$ real-valued elements along the diagonal. The elements
  in the upper half of the matrix are dependent on those in the lower half,
  hence the rest of the matrix has $1+2+\cdots+(n-1)$ free complex variables.
  Thus, $H(n)$ has dimension $2*[1+2+\cdots+(n-1)]+n=n^2$.

  Next, we wish to show that $I_n$ is a regular value of $f$:
  \begin{align*}
    df_A(h) &= \lim_{t\to0}\frac{f(A+th)-f(A)}{t}\\
            &= \lim_{t\to0}\frac{(A+th)(A+th)^*-AA^*}{t}\\
            &= \lim_{t\to0}\frac{AA^*+thA^*+tAh^*+t^2hh^*-AA^*}{t}\\
            &= \lim_{t\to0} hA^*+Ah^*+thh^*\\
            &= hA^*+Ah^*
  \end{align*}
  $I_n$ is then a regular value iff $df_A$ is surjective for all $A\in U(n)$.
  Let $C\in U(n)$, then $C=\frac{1}{2}C+\frac{1}{2}C^*$. Then, solving for $h$
  in $hA^*=\frac{1}{2}C$ yields $h=\frac{1}{2}CA$. This checks out, as
  \begin{align*}
    hA^*+Ah^* &= (\frac{1}{2}CA)A^*+A(\frac{1}{2}CA)^*\\
              &= \frac{1}{2}C(AA^*)+\frac{1}{2}A(A^*C^*)\\
              &= \frac{1}{2}CI_n+\frac{1}{2}I_nC\\
              &= C
  \end{align*}
  So $df_A$ is indeed surjective for all $A\in U(n)$.
  \hfill $\square$

\subsection*{Exercise A4}
  Let $U(1)$ be the unitary group and let $S^1$ be the circle group, then $f
  \colon U(1)\to S^1$ defined by $(a+ib)\mapsto (a,b)$ is a diffeomorphism.
  Suppose $x\in U(1)$, then $x\bar{x}=1$. Recall, for complex $z$, $\|z\|^2=z
  \bar{z}$. Therefore, $\|x\| = 1$, which implies $x\in S^1$. The reverse
  direction is no different.

  $SU(2)$ is the set $\{A\in U(2)\mid\det A = 1\}$. Suppose $A=
  \left(
    \begin{smallmatrix}
      a   &   b\\
      c   &   d
    \end{smallmatrix}
  \right)\in SU(2)$, then we have that $A^{-1}=\bar{A}^T$, or that
  $
  \left(
    \begin{smallmatrix}
      d           &   \text{-}b\\
      \text{-}c   &   a
    \end{smallmatrix}
  \right) =
  \left(
    \begin{smallmatrix}
      \bar{a}&\bar{c}\\
      \bar{b}&\bar{d}
    \end{smallmatrix}
  \right)
  $. Thus, $A$ must be of the form $
  \left(
    \begin{smallmatrix}
      a & \text{-}\bar{b}\\
      b & \bar{a}
    \end{smallmatrix}
  \right)
  $ for $a,b\in\C$. Since $\det A=1$, note that $a\bar{a}+b\bar{b}=1$.

  Next, consider the smooth maps $\sigma\colon\R^4\to\C^2$ and $\phi\colon\C^2
  \to M_2\C$ defined by $(x,y,z,w)\mapsto(x+iy, z+iw)$ and $(a,b)\mapsto
  \left(
    \begin{smallmatrix}
      a & \text{-}\bar{b}\\
      b & \bar{a}
    \end{smallmatrix}
  \right)$, respectively. Let $m=\phi\circ\sigma$.

  Suppose $s=(x,y,z,w)\in S^3$, then $x^2+y^2+z^2+w^2=1$. Denote $\sigma(s)$ as
  $(a,b)=(x+iy,z+iw)$. Recall, for $z=x+iy$, one has $z\bar{z}=x^2+y^2$. Hence,
  $a\bar{a}+b\bar{b}=1$, so $m(s)\in SU(2)$. Since $\sigma$ and $\phi$ are both
  smooth and invertible, $S^3$ is diffeomorphic to $SU(2)$ via $m$.
  \hfill $\square$

\end{document}
