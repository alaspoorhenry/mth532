\documentclass{article}
\usepackage[margin=1in]{geometry}
\usepackage{amsfonts}
\usepackage{amsmath}
\usepackage{amssymb}
\usepackage{parskip}
\usepackage{enumitem}
\usepackage{mathtools}

\newcommand{\N}{\mathbb{N}}
\newcommand{\Z}{\mathbb{Z}}
\newcommand{\Q}{\mathbb{Q}}
\newcommand{\R}{\mathbb{R}}
\newcommand{\C}{\mathbb{C}}
\DeclareMathOperator{\GL}{GL}

\begin{document}

\title{MTH 532 Homework 4}
\author{Roy Howie}
\date{February 16, 2017}
\maketitle

\subsection*{Exercise 7}
\subsection*{Exercise 9}
\subsection*{Exercise 12}
\subsection*{Exercise A1}
  Note that $M_n\C\cong\C^{n^2}$ and that $\det\colon M_n\C\to\C$ is a smooth
  function. Since $\C$ is Hausdorff, $0\in\C$ is closed. Likewise, because of
  the continuitity of $\det$, $\det^{-1}(0)$ is closed too. Note that $\GL_n\C=
  M_n\C-\det^{-1}(0)$, so $\GL_n\C$ is open and, therefore, a smooth manifold.

  Furthermore, $\GL_n\C$ is a group under matrix multiplication with identity
  $I_n$. Matrix multiplication is smooth, as it is a polynomial function in the
  entries of the product; matrix inversion is smooth because it is a rational
  function (Cramer's rule), as the determinant is non-vanishing in $\GL_n\C$.

  Hence, $\GL_n\C$ is a Lie group of dimension $2n^2$, as its basis consists of
  matrices with one element in $\{1,i\}$.
  \hfill $\square$

\subsection*{Exercise A2}
\subsection*{Exercise A3}
\subsection*{Exercise A4}
  Let $U(1)$ be the unitary group and let $S^1$ be the circle group, then $f
  \colon U(1)\to S^1$ defined by $(a+ib)\mapsto (a,b)$ is a diffeomorphism.
  Suppose $x\in U(1)$, then $x\bar{x}=1$. Recall, for complex $z$, $\|z\|^2=z
  \bar{z}$. Therefore, $\|x\| = 1$, which implies $x\in S^1$. The reverse
  direction is no different.

  $SU(2)$ is the set $\{A\in U(2)\mid\det A = 1\}$. Suppose $A=
  \left(
    \begin{smallmatrix}
      a   &   b\\
      c   &   d
    \end{smallmatrix}
  \right)\in SU(2)$, then we have that $A^{-1}=\bar{A}^T$, or that
  $
  \left(
    \begin{smallmatrix}
      d           &   \text{-}b\\
      \text{-}c   &   a
    \end{smallmatrix}
  \right) =
  \left(
    \begin{smallmatrix}
      \bar{a}&\bar{c}\\
      \bar{b}&\bar{d}
    \end{smallmatrix}
  \right)
  $. Thus, $A$ must be of the form $
  \left(
    \begin{smallmatrix}
      a & \text{-}\bar{b}\\
      b & \bar{a}
    \end{smallmatrix}
  \right)
  $ for $a,b\in\C$. Since $\det A=1$, note that $a\bar{a}+b\bar{b}=1$.

  Next, consider the smooth maps $\sigma\colon\R^4\to\C^2$ and $\phi\colon\C^2
  \to M_2\C$ defined by $(x,y,z,w)\mapsto(x+iy, z+iw)$ and $(a,b)\mapsto
  \left(
    \begin{smallmatrix}
      a & \text{-}\bar{b}\\
      b & \bar{a}
    \end{smallmatrix}
  \right)$, respectively. Let $m=\phi\circ\sigma$.

  Suppose $s=(x,y,z,w)\in S^3$, then $x^2+y^2+z^2+w^2=1$. Denote $\sigma(s)$ as
  $(a,b)=(x+iy,z+iw)$. Recall, for $z=x+iy$, one has $z\bar{z}=x^2+y^2$. Hence,
  $a\bar{a}+b\bar{b}=1$, so $m(s)\in SU(2)$. Since $\sigma$ and $\phi$ are both
  smooth and invertible, $S^3$ is diffeomorphic to $SU(2)$ via $m$.
  \hfill $\square$

\end{document}
